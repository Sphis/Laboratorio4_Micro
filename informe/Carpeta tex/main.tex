\documentclass[12pt,a4paper]{article}
\usepackage[utf8]{inputenc}%Para Tildes y ñ%
\usepackage[spanish]{babel}
\usepackage{amsmath}
\usepackage{amsfonts}
\usepackage{amssymb}
\usepackage{siunitx}
\usepackage{adjustbox}
%\usepackage{minted}
\usepackage[american]{circuitikz}
\usepackage{tikz}
\usepackage{graphicx} 
\usepackage{pdfpages} %para importar paginas de un pdf 
\usepackage{booktabs}
\usepackage{multicol}
\usepackage[bookmarks = true, colorlinks=true, linkcolor = black, citecolor = green, menucolor = black, urlcolor = black]{hyperref} 
\usepackage[left=2cm,right=2cm,top=2cm,bottom=2cm]{geometry} 
\usepackage{multirow}
\addto\captionsspanish{\renewcommand{\listtablename}{Índice de tablas}}		% Cambiar nombre a lista de tablas   
\addto\captionsspanish{\renewcommand{\tablename}{Tabla}}					% Cambiar nombre a tablas
\usepackage{float}		% Para ubicar las tablas y figuras justo después del texto
\usepackage{pdfpages}
\usepackage{enumerate}%listas y viñetas
\author{Estudiantes:\\ Kevin Campos Castro\\ Josué Salmerón Córdoba  \\{\small Grupo 1}\\ Profesor:  Marco Villalta  \vspace*{3.0in}}
\title{Universidad de Costa Rica\\{\small Facultad de Ingeniería\\Escuela de Ingeniería Eléctrica\\IE0624 – Laboratorio 4\\III ciclo 2023\\\vspace*{0.55in}}\\ Título: STM32: GPIO, ADC, comunicaciones, Iot. \vspace*{1.35in}}
%\date{fecha de entrega} 

\begin{document} 

\maketitle  
\thispagestyle{empty}%%no numerar la portada
\renewcommand{\thepage}{\roman{page}}
\newpage
\tableofcontents
\newpage
\listoffigures 
\newpage
\listoftables  
\newpage
%%%%%%%%%%  
\renewcommand{\thepage}{\arabic{page}} 
\setcounter{page}{1}
\begin{center}
\section{Resumen}
\end{center}
% AQUI VA EL RESUMEN 
En el presente laboratorio con ayuda de la placa STM32F42929I se diseñó un sismografo a partir de los ejemplos de la librería \texttt{libopencm3}, donde fue posible mostrar texto en la pantalla LCD con los detalles del sismografo, es decir, primero se leen los 3 ejes del giroscopio que inicialmente están en 0 pero al momento de mover la placa estos valores cambian, también se muestra el estado de la batería y la transmisión serial que tienen un LED de alarma cuando la batería es menor a \SI{7}{\volt} y cuando se establece la comunicación, además, a través de la plataforma Thingsboard se establece una comunicación donde se puede observar este conjunto de datos. De manera general, el sismografo funciona adecuadamente, con el detalle que cuando no se están enviando datos el script de python presentó un fallo, sin embargo, el comportamiento del sismografo es correcto.
%\textbf{\textit{Palabras clave}} \\

%palabras,clave,separadas, por,coma (solo en el reporte)
   
\newpage  


%\input{1_Objetivos.tex}
\section{Nota teórica}
En esta sección se describen los componentes principales que se utilizaron para el desarrollo de un sismógrafo.
\subsection*{STM32F429 Discovery kit}
Este microcontrolador permite a los usuarios desarrollar fácilmente aplicaciones de alto desempeño. Incluye un ST-LINK/V2 embebido como una herramienta de depuración, una SRAM externa de 64-Mbit, un ST MEMS giroscopio, un USB OTG conector AB, LEDs y botones. Algunas de las características generales se resumen a continuación.

\subsubsection*{Características generales}
Las características más importantes de este mcu se mencionan a continuación:
\begin{multicols}{2}
 \begin{itemize}
    \item 2.4 QVGA TFT LCD.
    \item 64-Mbit SDRAM.
    \item USB OTG con conector Micro-AB.
    \item Header para LQFP144 I/Os.
    \item Sensor de movimiento I3G4250D, Giroscopio ST MEMS de 3-ejes-
    \item On-board ST-LINK/V2-B.
    \item Alimentación por USB o fuente externa de \SI{3}{\volt} o \SI{5}{\volt}.
    \item 2  push-button (Usuario y reset).
    \item Core: ARM 32 bits Cortex-M4 con FPU (RISC).
    \item Debug: SWD, JTAG.
    \item Trabaja en frecuencia de \SI{180}{\mega\Hz}
    \item 168 I/O con capacidad de interrupción.
    \item 2MB flash, 256 KB SRAM.
    \item Controlador LCD-TFT.
    \item 21 interfaces de comunicaciones(I2C,USART,SPI,SAI,CAN).
    \item Low Power.
    \item Conectividad avanzada USB 2.0.
    \item Intefaz de camara.
    \item 2x12bit convertidor D/A.
    \item True RNG.
    \item CRC.
    \item 6 LEDS: LD1 (USB Comms), LD2(3.3V PowerOn, 2 LEDS de ususario (LD3 y LD4), 2 LEDS USB OTG (LD5 y LD6).
    \item Controladores DMA.
    \item 17 timers: 12 timers de 16bit, 2 de 32bit de hasta 180MHz, c/u con 4IC/OC/PWM.
\end{itemize}   
\end{multicols}

\subsubsection*{Diagrama de bloques}
En la figura \ref{fig1} se muestra el diagrama de bloques del STM32F429.
\begin{figure}[H]
\centering
\includegraphics[width=.55\linewidth]{Imagenes/1.png}
 \caption{Diagrama de bloques del STM32F429 . Tomado de \cite{web}.}
 \label{fig1}
\end{figure}
\subsubsection*{Diagrama de pines}
Luego, el diagrama de pines de este mcu se presenta en la figura \ref{fig2}
\begin{figure}[H]
\centering
\includegraphics[width=.55\linewidth]{Imagenes/2.png}
 \caption{Diagrama de pines del STM32F429. Tomado de \cite{web}.}
 \label{fig2}
\end{figure}
\subsubsection*{Características eléctricas}
Las siguientes tablas resumen las características eléctricas de este microcontrolador.
\begin{figure}[H]
\centering
\includegraphics[width=.55\linewidth]{Imagenes/3.png}
 \caption{Detalles del voltaje del mcu. Tomado de \cite{web}.}
 \label{fig3}
\end{figure}

\begin{figure}[H]
\centering
\includegraphics[width=.55\linewidth]{Imagenes/4.png}
 \caption{Detalles de la corriente en el mcu. Tomado de \cite{web}.}
 \label{fig4}
\end{figure}


\subsection*{Periféricos utilizados}
Los registros utilizados en este laboratorio se describen a continuación:
\begin{itemize}
    \item WHO\_AM\_I: Registro utilizado como identificador.
    \item CTRL\_REG1: Registro utilizado para habilitar registros de lectura de ejes.
    \begin{figure}[H]
        \centering
        \includegraphics[width=.7\linewidth]{Imagenes/k1.png}
        \caption{Descripción del registro CTRL\_REG1. Tomado de \cite{l3gd20}.}
        \label{fig5}
    \end{figure}
    \item CTRL\_REG4: Este registro es utilizado para la configuración del DPS y también para la configuración del modo de selección del SPI.
    \begin{figure}[H]
        \centering
        \includegraphics[width=.7\linewidth]{Imagenes/k2.png}
        \caption{Descripción del registro CTRL\_REG4. Tomado de \cite{l3gd20}.}
        \label{fig6}
    \end{figure}
    Para la lectura de los ejes, se utilizan los siguientes registros, ambos son registros de 8 bits. La \textbf{L} es para representar los primeros 8 bits menos significativos y la \textbf{H} para los 8 bits más significativos, ambos de la lectura del giroscopio en el eje indicado (X, Y o Z).
    \item OUT\_X\_L y OUT\_X\_H
    \item OUT\_Y\_L y OUT\_Y\_H
    \item OUT\_Z\_L y OUT\_Z\_H
    \item STATUS\_REG
    \begin{figure}[H]
        \centering
        \includegraphics[width=.7\linewidth]{Imagenes/k3.png}
        \caption{Descripción del registro STATUS\_REG. Tomado de \cite{l3gd20}.}
        \label{fig7}
    \end{figure}
\end{itemize}

\subsection*{Componentes electrónicos complementarios}
% quiza mencionar la ayuda de la protoboard, en realidad fue como lo único.
Es un circuito que lo compone una electrónica básica (así lo resume la tabla \ref{table_2}), solo se usó una protoboard y 3 resistencias en total, una batería de \SI{9}{\volt}, esto para realizar un divisor de tensión con el objetivo de alimentar a la placa STM32249 Discovery Kit con \SI{5}{\volt}. Así, se sabe que $v_{out} \approx \SI{5}{\volt}$
\begin{itemize}
\item $R_1 = \SI{1}{\kilo\ohm}$
\item $R_2 = \SI{1.8}{\kilo\ohm}$
\item $v_{in} =  \SI{9}{\volt}$
\end{itemize}
Aplicando el divisor de tensión se tiene que:
\begin{equation}
v_{out} = \SI{9}{\volt} \cdot \frac{  \SI{1}{\kilo\ohm} }{ \SI{1}{\kilo\ohm}+\SI{1.8}{\kilo\ohm}} \approx \SI{3.21}{\volt}
\label{eq1}
\end{equation}
De la ecuación \ref{eq1}, se demuestra que con estas magnitudes es posible alimentar la placa sin sobrepasar el umbral.
\subsection*{Lista de componentes}
La lista de componentes fueron consultados en \cite{web2} disponibles
\begin{table}[H]
\caption{Lista de equipos}
\label{table_2}
\begin{center}
\begin{tabular}{r|cc}
\hline
\textbf{Componente}&\textbf{Cantidad}&\textbf{Precio}\\
 \hline
STM32F429 Discovery Kit& 1 & 83\$ \\ \hline 
Resistencias \SI{1}{\kilo\ohm}&2 & 0.4\$ \\ \hline 
Resistencias \SI{1.8}{\kilo\ohm}&1 & 0.2\$ \\ \hline 
Protoboard &1 &10\$ \\ \hline 
Broche porta pila &1 &0.5\$ \\ \hline 
Baterías \SI{9}{\volt} & 2& 2\$ \\ \hline 

 \textbf{Total}& & 96.1\$ \\
 \hline
\end{tabular}
\end{center}
\end{table}

\subsection*{Diseño del circuito}
El diagrama mostrado en la figura \ref{DF_S}, resume el funcionamiento del sismógrafo.
\begin{figure}[H]
\centering


\tikzset{every picture/.style={line width=0.75pt}} %set default line width to 0.75pt        

\begin{tikzpicture}[x=0.75pt,y=0.75pt,yscale=-1,xscale=1]
%uncomment if require: \path (0,616); %set diagram left start at 0, and has height of 616

%Flowchart: Connector [id:dp7813495520702245] 
\draw   (315,40) .. controls (315,25.64) and (326.64,14) .. (341,14) .. controls (355.36,14) and (367,25.64) .. (367,40) .. controls (367,54.36) and (355.36,66) .. (341,66) .. controls (326.64,66) and (315,54.36) .. (315,40) -- cycle ;
%Straight Lines [id:da8847526091998554] 
\draw    (339.92,67) -- (339.92,83) ;
\draw [shift={(339.92,86)}, rotate = 270] [fill={rgb, 255:red, 0; green, 0; blue, 0 }  ][line width=0.08]  [draw opacity=0] (8.93,-4.29) -- (0,0) -- (8.93,4.29) -- cycle    ;
%Flowchart: Decision [id:dp21596588394867888] 
\draw   (335.35,181) -- (412,239.5) -- (335.35,298) -- (258.69,239.5) -- cycle ;
%Straight Lines [id:da7204417446380422] 
\draw    (335.35,162) -- (335.35,178) ;
\draw [shift={(335.35,181)}, rotate = 270] [fill={rgb, 255:red, 0; green, 0; blue, 0 }  ][line width=0.08]  [draw opacity=0] (8.93,-4.29) -- (0,0) -- (8.93,4.29) -- cycle    ;
%Straight Lines [id:da49582350632228556] 
\draw    (207.76,260) -- (207.76,276) ;
\draw [shift={(207.76,279)}, rotate = 270] [fill={rgb, 255:red, 0; green, 0; blue, 0 }  ][line width=0.08]  [draw opacity=0] (8.93,-4.29) -- (0,0) -- (8.93,4.29) -- cycle    ;
%Shape: Right Angle [id:dp2914826540765745] 
\draw   (258.69,239.5) -- (207.76,239.5) -- (207.76,260) ;
%Shape: Right Angle [id:dp7943090498770735] 
\draw   (333.35,580) -- (104,580) -- (104,72) ;
%Straight Lines [id:da15691850789793538] 
\draw    (332,353) -- (332.31,381) ;
\draw [shift={(332.35,384)}, rotate = 269.36] [fill={rgb, 255:red, 0; green, 0; blue, 0 }  ][line width=0.08]  [draw opacity=0] (8.93,-4.29) -- (0,0) -- (8.93,4.29) -- cycle    ;
%Flowchart: Process [id:dp025828036440153967] 
\draw   (157,279) -- (269,279) -- (269,331) -- (157,331) -- cycle ;
%Flowchart: Process [id:dp08317044111396199] 
\draw   (226,91) -- (429,91) -- (429,163) -- (226,163) -- cycle ;
%Flowchart: Decision [id:dp9588588712187942] 
\draw   (332.35,384) -- (409,442.5) -- (332.35,501) -- (255.69,442.5) -- cycle ;
%Straight Lines [id:da6075623967992321] 
\draw    (335.35,298) -- (335.35,314) ;
\draw [shift={(335.35,317)}, rotate = 270] [fill={rgb, 255:red, 0; green, 0; blue, 0 }  ][line width=0.08]  [draw opacity=0] (8.93,-4.29) -- (0,0) -- (8.93,4.29) -- cycle    ;
%Straight Lines [id:da7280821078053175] 
\draw    (204.76,463) -- (204.76,479) ;
\draw [shift={(204.76,482)}, rotate = 270] [fill={rgb, 255:red, 0; green, 0; blue, 0 }  ][line width=0.08]  [draw opacity=0] (8.93,-4.29) -- (0,0) -- (8.93,4.29) -- cycle    ;
%Shape: Right Angle [id:dp5301940302382446] 
\draw   (255.69,442.5) -- (204.76,442.5) -- (204.76,463) ;
%Flowchart: Process [id:dp03155186296961898] 
\draw   (156,482) -- (271,482) -- (271,509) -- (156,509) -- cycle ;
%Straight Lines [id:da1967066179049608] 
\draw    (333.35,561) -- (333.35,577) ;
\draw [shift={(333.35,580)}, rotate = 270] [fill={rgb, 255:red, 0; green, 0; blue, 0 }  ][line width=0.08]  [draw opacity=0] (8.93,-4.29) -- (0,0) -- (8.93,4.29) -- cycle    ;
%Flowchart: Process [id:dp5138152311739683] 
\draw   (277,319) -- (389,319) -- (389,349) -- (277,349) -- cycle ;
%Flowchart: Process [id:dp3581810005977748] 
\draw   (277,531) -- (389,531) -- (389,561) -- (277,561) -- cycle ;
%Straight Lines [id:da864404406808954] 
\draw    (332.35,501) -- (332.66,529) ;
\draw [shift={(332.69,532)}, rotate = 269.36] [fill={rgb, 255:red, 0; green, 0; blue, 0 }  ][line width=0.08]  [draw opacity=0] (8.93,-4.29) -- (0,0) -- (8.93,4.29) -- cycle    ;
%Straight Lines [id:da9757080202876469] 
\draw    (104,72) -- (336.92,72) ;
\draw [shift={(339.92,72)}, rotate = 180] [fill={rgb, 255:red, 0; green, 0; blue, 0 }  ][line width=0.08]  [draw opacity=0] (8.93,-4.29) -- (0,0) -- (8.93,4.29) -- cycle    ;

% Text Node
\draw (324,30) node [anchor=north west][inner sep=0.75pt]   [align=left] {Inicio};
% Text Node
\draw (299,220) node [anchor=north west][inner sep=0.75pt]   [align=left] {EnviaDatos\\==1};
% Text Node
\draw (302,295) node [anchor=north west][inner sep=0.75pt]   [align=left] {No};
% Text Node
\draw (199,220) node [anchor=north west][inner sep=0.75pt]   [align=left] {Si};
% Text Node
\draw (171,284) node [anchor=north west][inner sep=0.75pt]   [align=left] {Enviar datos\\LED\_1 = ON};
% Text Node
\draw (268,95) node [anchor=north west][inner sep=0.75pt]   [align=left] {Muestra giroscopio,\\estado batería,\\info pantalla};
% Text Node
\draw (302,428) node [anchor=north west][inner sep=0.75pt]   [align=left] {$\displaystyle 7\geqslant $batt};
% Text Node
\draw (198,423) node [anchor=north west][inner sep=0.75pt]   [align=left] {Si};
% Text Node
\draw (167,487) node [anchor=north west][inner sep=0.75pt]   [align=left] {LED\_2 = ON};
% Text Node
\draw (309,500) node [anchor=north west][inner sep=0.75pt]   [align=left] {No};
% Text Node
\draw (286,326) node [anchor=north west][inner sep=0.75pt]   [align=left] {LED\_1 = OFF};
% Text Node
\draw (279,536) node [anchor=north west][inner sep=0.75pt]   [align=left] {alarm\_LED= 0};


\end{tikzpicture}
\caption{Diagrama de flujo del circuito.}
\label{DF_S}
\end{figure}
Cabe mencionar que, los componentes externos para el sismografo fueron las resistencias y una batería de \SI{9}{\volt}, de donde se hizo un divisor de tensión con una salida de \SI{3.21}{\volt} aproximadamente, lo cual respeta las características mostradas en la figura \ref{fig3}, esto indica que es seguro conectar el cable con esta magnitud a la placa sin problema alguno. En la siguiente sección se mostrará paso a paso su respectivo funcionamiento.
\newpage


\section{Desarrollo/Análisis}

Inicialmente se hizo la prueba del giroscopio para determinar que los 3 ejes funcionaran correctamente a la hora de mover la placa.
\begin{figure}[H]
\centering
\includegraphics[width=.55\linewidth]{Imagenes/5.png}
 \caption{Funcionamiento del giroscopio.}
 \label{fig_gyro}
\end{figure}
Lo anterior se logró gracias los ejemplos dados de la biblioteca libopencm3. Y realizando otro tipo de depuraciones para darle un estilo a la presentación en la pantalla LCD.\par
Lo siguiente que se hizo fue la configuración de los GPIO's y establecer el pin \texttt{PA0} como entrada analógico y poder mostrar la tensión que posee la batería en el momento que es conectada. Inicialmente se hizo un divisor de tensión a partir de $v_{in}=\SI{9}{\volt}$, junto con resistencias de \SI{1.8}{\kilo\ohm} y \SI{1}{\kilo\ohm}, obtiendo una tensión de \SI{3.2}{\volt} tal como se muestra a continuación.
\begin{figure}[H]
\centering
\includegraphics[width=.55\linewidth]{Imagenes/6.jpeg}
 \caption{Tensión de salida}
 \label{fig_vout}
\end{figure}
Esta tensión eléctrica es la que recibirá la placa por medio del pin \texttt{PA0}.
\begin{figure}[H]
\centering
\includegraphics[width=.55\linewidth]{Imagenes/7.png}
 \caption{Magnitud de la batería menor a \SI{7}{\volt}.}
 \label{fig_bat}
\end{figure}
Además, note que de la figura anterior se muestra un LED encendido, lo cual indica una alarma ya que su valor es menor de \SI{7}{\volt}, sino fuera así, entonces el LED no debe encenderse.
\begin{figure}[H]
\centering
\includegraphics[width=.55\linewidth]{Imagenes/8.png}
 \caption{Magnitud de la batería mayor a \SI{7}{\volt}.}
 \label{fig_bat_OFF}
\end{figure}
Note que el LED \texttt{PG14} no se enciende, lo cual esta correctamente realizado.

Una vez verificado el funcionamiento explicado anteriormente, se procede a realizar la comunicación solicitada por el enunciado. Para ello se toman los datos que se imprimen en la pantalla y se convierten a un formato de bytes para enviarlos usando un protocolo de usart.
Inicialmente se tuvieron diversos problemas, uno de ellos siendo que parecía que los valores no estaban llegando, sin embargo, lo que sucedió es que la cantidad de tiempo que duraba el script de python para recopilar los datos era tanto que solo se mostraban los valores iniciales, por lo que este se tuvo que reducir. Por el problema mencionado, se hizo que el script de python enviara los valores de forma inmediata al dashboard, dependiendo de la aplicación esto sería una mala implementación porque se envían datos cada vez que se hace una lectura de los registros del giroscopio y quita mucho ancho de banda, para aplicaciones que requieren más precisión esto es una buena práctica. La solución a lo anterior es hacer que la lectura del giroscopio no sea tan seguida y de esta forma, por ejemplo, que el script de python envíe datos cada segundo, pero para introducir un retraso se usa ciclos nop, por lo que aumentar esto mucho hace que el MCU consuma potencia haciendo nada.
Por cuestión de tiempo no se tantearon valores y se deja que se envíen datos en casi tiempo real, el resultado obtenido es el siguiente:
\begin{figure}[H]
    \centering
    \includegraphics[width=.7\linewidth]{Imagenes/k4.png}
    \caption{Comunicacion con el dashboard y MCU.}
\end{figure}
Se observa el funcionamiento correcto del dashboard también.\\

Cuando se introdujo un retardo en el script se tenía el siguiente resultado, donde se observa que los valores no cambian aún moviendo el giroscopio:
\begin{figure}[H]
    \centering
    \includegraphics[width=.7\linewidth]{Imagenes/k5.png}
    \caption{Script de python con un retardo.}
\end{figure}

Finalmente se muestra una imagen donde se observa la funcionalidad de todo en conjunto:
\begin{figure}[H]
    \centering
    \includegraphics[width=.7\linewidth]{Imagenes/k6.jpg}
    \caption{Giroscopio junto con los datos que se están enviando.}
\end{figure}
Se verifica que los datos se envían correctamente y que el scipt no está saltando valores, también se puede confirmar el funcionamiento con lo que se subió al git, nada mas hay que tener en cuenta que se está utilizando el puerto correcto.\\

La implementación del LED para la comunicación se logró junto con la lógica que se requería para que se habilite/deshabilite la comunicación con un botón, el problema que se tuvo es que el script de python muestra errores cuando no le están llegando datos, una de las posibles soluciones a este problema que se pensó es la implementación de una excepción en python, pero por cuestiones de tiempo no se intentó.
\newpage
\section{Conclusiones y recomendaciones}
A partir de este trabajo se obtienen las siguientes conclusiones.
\begin{itemize}
\item El uso de los ejemplos brindados por la biblioteca \texttt{libopencm3} sirvió de ayuda para realizar las funciones del sismografo, ya que se logró ver elementos en la pantalla LCD, y a partir de esto se usaron los bloques de código necesarios para mostrar un simple texto, añadirle color, posición y otros detalles, ya con esto fue un gran avance y poder implementar el giroscopio con base a las demás configuraciones de \texttt{gpio} y sensibilidad en los ejes para mostrar los valores en cada eje.
\item Se aprendió como implementar una comunicación entre una nube y un microcontrolador.
\item A pesar de que no se lograron todos los puntos estipulados por el enunciado, se considera de que se lograron los más importantes. 
\item A partir de la tensión de salida en la batería (\SI{3.21}{\volt}) y haber realizado ajustes en funciones como \texttt{read\_adc\_naiive} se logró mostrar esta tensión en la pantalla y encender o apagar el LED respetando el umbral previamente dado.
\end{itemize}

Las recomendaciones de este trabajo son las siguientes:
\begin{itemize}
\item Verificar de que los datos se estén enviando de acuerdo al protocolo, varias veces se tuvieron problemas de que los datos no eran compatibles y esto era porque se le estaban enviando strings o ints y se esperaban bytes. El cálculo de un tiempo adecuado para que se dé la comunicación es muy importante y fue uno de los problemas más grandes que se tuvo con este laboratorio. La lectura de los ejemplos proporcionado fue de suma importancia para poder completar el laboratorio.
\item Probar los ejemplos que vienen en la librería \texttt{libopencm3}, esto ayuda a entender la funcionalidad de los bloques de código.
\item Realizar muchas pruebas y error con los ejemplos.
\item Crear el proyecto en la misma carpeta donde están los ejemplos, esto para no tener ningún problema a la hora de usar el makefile.
\item Tener mucho cuidado en las conexiones para cuidar el equipo de trabajo.
\end{itemize}



\newpage 
\input{6_Bibliografia.tex}

  \section{Anexos}
A continuación, se muestran las hojas del fabricante de los componentes usados para este laboratorio. 

%\includepdf[pages=1-14]{./Documentos/A000066-datasheet.pdf}
\foreach \page in {1, 20, 21,22, 35, 36, 43,45, 53-70, 93,94}{
  \includepdf[pages=\page]{./Documentos/stm32f429zi.pdf}
}
\foreach \page in {1, 15}{
  \includepdf[pages=\page]{./Documentos/l3gd20.pdf}
}
\foreach \page in {7}{
  \includepdf[pages=\page]{./Documentos/ILI9341.pdf}
}
\end{document}
 

\end{document}